% 模板定义了imuthesis宏包,派生自ctexbook,依赖ctex
\documentclass[hyperref]{imuthesis}

\begin{document}

	%个人信息注入
	\setThesisTitle{内蒙古大学~本科学位论文~\LaTeX模板}
	\setThesisTitleEng{The IMU Dissertation  Template of {\LaTeX}}
	\setStudentId{12345678}
	\setCLC{TP39}
	\setCollege{计算机学院}
	\setMajor{计算机科学与技术}
	\setGrade{2010级}
	\setAuthor{石博天}
	\setAuthorEng{Botian, Shi}
	\setTutor{张学良}
	\setTutorEng{Xueliang, Zhang}
	\setUniversityId{10126}
	\setYear{2014}
	\setMonth{4}
	\setDate{11}
	\setKeywords{{\LaTeX},论文,模板,内蒙古大学}
	\setKeywordsEng{{\LaTeX}, Dissertation, Template, IMU}
	
	\maketitle
	
	\cleardoublepage
	\begin{cabstract}	%摘要正文文本前一定不要空行!摘要写完后一定需要空行!
		本文介绍了~\emph{imuthesis}~这个文档模板所提供的功能,
		并以自身为例演示了该模板的使用。
		论文摘要应概括地反映出毕业论文(设计)的目的、内容、方法、成果和结论。
		摘要中不宜使用公式、图表,不标注引用文献编号。
		摘要以300~500字为宜(外文摘要与中文摘要相对应)。

	\end{cabstract}
	
	\cleardoublepage
	\begin{eabstract}
		This paper describes the the functions provided by the imuthesis document template, 
		and provides itself as an example to illustrate the usage of the document class.

	\end{eabstract}
	
\end{document}

