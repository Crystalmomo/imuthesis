\chapter{绪论}
    本文档是\emph{内蒙古大学本科学位论文\LaTeX模板——imuthesis}的使用说明。本说明自身即使用imuthesis编写。文档版本:V/0.91。构建于\today。你可以在\url{https://github.com/friskit-china/imuthesis}上找到本文档的全部源代码。

    模板格式符合内蒙古大学本科学文论文规范,对字号、段间距、标题格式、目录格式、参考文献、图片、代码等多个方面做了格式化工作。并且支持多文件结构、BibTeX文献管理工具、lstlisting代码高亮优化等模块。在编写过程大量参考了\emph{pkuthss}。

    模板于\TeX Live 2013环境下开发,以\CTeX宏包为中文化基础,目前支持使用XeLaTeX编译。

    以下是模板所包含的\emph{有用文件}:
    \begin{quote}
        \begin{description}
            \item [imuthesis.cls] 文档类型文件,定义了绝大部分模板的命令、格式等。
            \item [paper.tex] 本文档
            \item [paper.bib] BibTex文档
            \item [chinesebst.bst] BibTex Style文档
            \item [chapters文件夹] 用来保存各章.tex文件
            \item [codes文件夹] 用来保存代码文件
            \item [figure文件夹] 用来保存图片文件
            \item [img文件夹] 本模板所需的图片文件(修改)。 
        \end{description}
    \end{quote}
    在接下来的几章中,将会为大家展示此模板的各种格式说明与使用介绍。

